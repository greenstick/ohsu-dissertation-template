\center

\section{Introduction}

\raggedright

The purpose of the introduction is to elucidate the nature of the problem
addressed by the thesis or dissertation research. The problem should be clearly presented, and its history discussed through a survey of the literature.
The author should explain the rationale behind the scientific approach to the problem. This section typically opens with a high level introduction of the topic and closes with the statement of the project aims (e.g. 3 specific aims) and their associated hypotheses.

There maybe be several citations. To reference a citation, call the citation key. For example, \cite{OHSU-overleaf-dissertation-template}. See the references.bib for help inferring how this works (the referenced citation key needs to match the citation key in references.bib \textit{exactly}). Multiple citations can be stringed together by simply separating them with commas (see the comment below). 

% \cite{ref1, ref2, ref3}

\subsection{Some subsection}
\lipsum
