\center

\section{Background}

\raggedright

This section should include a comprehensive review of prior work across the stated aims. It also provides a summary of the gaps in the current literature and is will be substantially longer than a background section in a manuscript.

%%
%% Citations
%%

\subsection{Referencing Citations}

Citations are straight forward and will be automatically sorted on rendering. Include your citations in the \textit{references.bib} file. For examples, see this  \textit{\href{https://libguides.nps.edu/citation/ieee-bibtex}{handy citation guide}}. Here's an example usage where I cite this project \cite{OHSU-overleaf-dissertation-template}. Boom!

%%
%% Including Figures
%%

\subsection{Including figures}

You might include a figure here...

\begin{figure*}
  \centering
  \includegraphics[width=\textwidth]{"figures/thesis-defense".png}
  \caption{\textbf{The best thesis defense is a good thesis offense.} A conceptual illustration of the celebrated thesis \textit{offense}, an ambitious but often effective tactical maneuver.}
  \label{fig:fig1}
\end{figure*}

...and reference it like so: \textbf{Figure \ref{fig:fig1}}.

%%
%% Making Tables
%%

\subsection{Making tables}

Or maybe you'll make a table...

\begin{table}[h!]
  \begin{center}
    \caption{Your first table.}
    \label{tab:table1}
    \begin{tabular}{l|c|r} % <-- Alignments: 1st column left, 2nd middle and 3rd right, with vertical lines in between
      \textbf{Value 1} & \textbf{Value 2} & \textbf{Value 3}\\
      $\alpha$ & $\beta$ & $\gamma$ \\
      \hline
      1 & 1110.1 & a\\
      2 & 10.1 & b\\
      3 & 23.113231 & c\\
    \end{tabular}
    \label{table:table1}
  \end{center}
\end{table}

...an reference it too: \textbf{Table \ref{table:table1}}

%%
%% Using Abbreviations
%%

\subsection{Using Abbreviations}

You may also use abbreviations like \acrfull{QML} or \acrfull{NISQ} device. Make sure you define the abbreviations in the \textit{glossary.tex} file. Further, you can use different commands. For the first usage in a section use \verb|\acrfull| to show the full wording followed by the acronym in parentheses, as seen above. Use \verb|\acrlong| to show just the full wording (e.g. \acrlong{QIP} – this can be good for instances where the acronym might lead to ambiguity) or \verb|\acrshort| for just the acroynm itself (e.g. \acrshort{FTQC}, which stands for the fault tolerant quantum computing paradigm!)